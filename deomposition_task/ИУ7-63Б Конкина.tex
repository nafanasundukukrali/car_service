\documentclass{bmstu}

\usepackage{csquotes}
\usepackage[utf8]{inputenc}
\usepackage[T1]{fontenc}


\NewBibliographyString{accessmode}
\DeclareFieldFormat{url}{\bibstring{accessmode}\addcolon\space\url{#1}}
\DeclareFieldFormat{title}{#1}
\DefineBibliographyStrings{russian}{
	accessmode={Режим доступа:},
	urlseen={дата обращения:}
}


\setenumerate[1]{label={ \arabic*)}}
\usepackage{enumitem}

\usepackage{placeins}
\usepackage{cleveref}
\usepackage{fancyvrb}

\newcommand{\dlr}{\mbox{\textdollar}}


\renewcommand\labelitemi{---}


\usepackage{pgffor}

\usepackage{longtable}

\usepackage{tabularx,booktabs,caption,ragged2e}
\newcolumntype{L}{>{\RaggedRight\arraybackslash}X}



%\captionsetup[table]{justification=raggedleft,singlelinecheck=off}

\renewcommand{\includelistingpretty}[3]
{
	\lstinputlisting[
	language={#2},
	keywordstyle=\color{keywords},
	stringstyle=\color{strings},
	commentstyle=\color{comments},
	frame=single,
	numbers=left,
	extendedchars=true,
	breaklines=true,
	numberstyle=\footnotesize\color{numbers},
	caption={#3},
	label={lst:#1},
	]{inc/lst/#1}
}

\usepackage{listings}

\lstdefinestyle{mystyle}{
	keywordstyle=\color{keywords},
	stringstyle=\color{strings},
	commentstyle=\color{comments},
	frame=single,
	numbers=left,
	extendedchars=true,
	breaklines=true,
	numberstyle=\footnotesize\color{numbers}
}

\usepackage{caption}
\usepackage{chngcntr}


\newcommand{\specialcell}[2][c]{%
	\begin{tabular}[#1]{@{}c@{}}#2\end{tabular}}
	
	
\titleformat{\chapter}[block]{\hspace{\parindent}\large\bfseries}{\thechapter}{0.5em}{\large\bfseries\raggedright}
\titleformat{name=\chapter,numberless}[block]{\hspace{\parindent}}{}{-50pt}{\large\bfseries\centering}
\titleformat{\section}[block]{\hspace{\parindent}\large\bfseries}{\thesection}{0.5em}{\large\bfseries\raggedright}
\titleformat{\subsection}[block]{\hspace{\parindent}\large\bfseries}{\thesubsection}{0.5em}{\large\bfseries\raggedright}
\titleformat{\subsubsection}[block]{\hspace{\parindent}\large\bfseries}{\thesubsection}{0.5em}{\large\bfseries\raggedright}
\titlespacing{\chapter}{12.5mm}{-20pt}{20pt}
\titlespacing{\section}{12.5mm}{10pt}{10pt}
\titlespacing{\subsection}{12.5mm}{10pt}{10pt}
\titlespacing{\subsubsection}{12.5mm}{10pt}{10pt}

\usepackage{siunitx}
\usepackage{hyperref}
\usepackage{nameref}

\usepackage{sverb}

%\makeatletter
%\def\verbatim{\footnotesize\@verbatim\small %\frenchspacing\@vobeyspaces \@xverbatim}
%\makeatother

	
\usepackage{verbatim}
\newenvironment{myverbatim}%
{\endgraf\noindent MYVERBATIM:\endgraf\verbatim}%
{\endverbatim}

\makeatletter
\newenvironment{verbatimlisting}[1]%
{\def\verbatim@startline{\input{#1}%
		\def\verbatim@startline{\verbatim@line{}}%
		\verbatim@startline}%
\verbatim}{\endverbatim}

\newwrite\verbatim@out

\makeatother

\usepackage{listings}

\usepackage{array, makecell}


\usepackage{svg}
\usepackage{pdfpages}

\newcommand{\myincludesvg}[5]
{
	\ifthenelse{\equal{#2}{f}}
	{
		\begin{figure}[#3]
			\center{\includesvg[width=#4]{inc/img/#1}}
			\caption{#5}
			\label{img:#1}
		\end{figure}
	}
	{
		\ifthenelse{\equal{#2}{w}}
		{
			\begin{wrapfigure}{#3}{#4}
				\center{\includesvg[width=#4]{inc/img/#1}}
				\caption{#5}
				\label{img:#1}
			\end{wrapfigure}
		}
		{
			\PackageError{bmstu}{unknown image type}{}	
		}
	}
}

\bibliography{./biblio.bib}


\begin{document}
	Целью курсовой работы является разработка базы данных заданной предметной области.
	
	Для достижения поставленной цели необходимо решить следующие задачи:
	\begin{itemize}
		\item проанализировать предметную область область, рассмотреть существующие решения;
		\item спроектировать программное обеспечение для реализации базы данных;
		\item выбрать средства реализации программного обеспечения;
		\item Исследовать зависимости времени выполнения запроса от объема обрабатываемых данных.
	\end{itemize}
	
	Выделены следующие этапы выполнения курсовой работы:
	\begin{itemize}
		\item уточнение технического задания;
		\item этап анализа;
		\item этап проектирования;
		\item технологический этап;
		\item проведение исследования;
		\item создание презентации;
		\item защита курсовой работы.
	\end{itemize}
	
	На рисунке \ref{img:gant} приведена на диаграмме Ганта планируемая модель выполнения курсовой работы по каждой учебной неделе.
	
	\includeimage
	{gant} 
	{f} 
	{H} 
	{0.9\textwidth} 
	{диаграмма Ганта планируемой модели выполнения курсовой работы по каждой учебной неделе}
	
	Результатом этапа уточнения технического задания является техническое задание.
	
	Результатом этапа анализа являются следующие артефакты:
	\begin{itemize}
		\item таблица сравнений существующих решений на основании заранее выбранных критериев;
		\item ER-диаграмма сущностей в нотации Чена;
		\item диаграмма прецедентов использования приложения;
		\item таблица сравнения существующих моделей баз данных на основании заранее выбранных критериев.
	\end{itemize}
	
	Результатом этапа проектирования являются следующие артефакты:
	\begin{itemize}
		\item диаграмма классов разрабатываемого ПО;
		\item ER-диаграмма описания сущностей базы данных;
		\item схемы алгоритмов триггеров.
	\end{itemize}
	
	Результатом технологического этапа являются следующие артефакты:
	\begin{itemize}
		\item таблица сравнения СУБД на основании заранее выбранных критериев;
		\item сценарии функционального и модульного тестирования;
		\item листинги кода;
	\end{itemize}
	
	Результатом проведения исследования являются следующие артефакты:
	\begin{itemize}
		\item сводная таблица результатов тестирования;
		\item таблица результатов исследования;
		\item график на основании таблицы результатов исследования.
	\end{itemize}
	
	
	
	
\end{document}
