\chapter*{ВВЕДЕНИЕ}
\addcontentsline{toc}{chapter}{ВВЕДЕНИЕ}

Уровень автомобилизации в стране -- один из индикаторов благосостояния ее граждан\cite{diler}. 
Согласно данным АЕС\cite{stat2022aebrus}, автомобильный рынок в России в 2022 году сократился на $58,8\%$, к концу 2022 года из 60 брендов, работавших в начале того же года, остались работать 14, 3 из которых принадлежали российским компаниям. 
В 2022-2023 годах 43 бренда вошли в список параллельного импорта Минпромторга\cite{parallelimport}, благодаря чему, согласно данным АЕС за 2023 год\cite{stat2023aebrus}, несмотря на нулевую долю рынка, 3\% проданных новых автомобилей приходились на бренды, не оказывающих поддержку в России, а, в совокупности, автомобильный рынок в 2023 году вырос на 57,8\% по сравнению с 2022 годом. 

В связи с ростом темпов автомобилизации возрастает потребность в поддержании техники в исправном состоянии\cite{diler}. 
В свою очередь, это сопровождается увеличением роли автосервисных предприятий, выполняющих такие услуги\cite{diler}.
 
Сервис является функцией, создающей потребительскую ценность, и является одним из элементов дифференциации компании на рынке\cite{upravavto}. 
Рынок автомобильных услуг -- это отношения между субъектами этого рынка: автовладельцами и предприятия системы автосервиса\cite{upravavto}.
Конечная цель сервиса -- развитие и поддержание взаимовыгодных отношений со стратегически важными клиентами и создание лояльных клиентов посредством формирования восприятия клиентами высокой потребительской ценности продукта компании\cite{upravavto}. 
Для установления и поддержания долгосрочных отношений с клиентами компания должна знать каждого клиента, концентрация данных о клиентах из всех каналах и точках взаимодействия в базе данных является основой аналитического процесса\cite{upravavto}. 

Согласно данным агентства <<Автостат>> \cite{yearold}, средний возраст легковых автомобилей в РФ по состоянию на 1 января 2023 года составил 14,7 года. 
При этом, основные затраты на ремонт автомобиля приходятся на вторую половину этого срока, а для сервисного рынка всех стран характерна общая картина -- заказчики, которые купили у официального дилера, предпочитают обращаться в независимые ремонтные фирмы и мелкие специализированные мастерские\cite{volgin}. 
Однако, поскольку обслуживание автомобилей, входящих в список товаров, подлежащему параллельному импорту, никоим образом не может включать гарантийный ремонт на основании договора с фирмой-продавцом, то обсуждать какое-либо официальное обслуживание автомобилей подобных брендов не имеет смысла, а компании, до 2022 года являвшиеся официальными дилерами, можно соотнести с прочими независимыми компаниями, предоставляющими услуги ремонта, что говорит о росте конкуренции в данном сегменте. 

Целью курсового проекта является разработка базы данных автомобильного сервиса, специализирующегося на брендах, не оказывающих поддержку в России.

Для достижения поставленной цели необходимо выполнить следующие задачи:

\begin{enumerate}
	\item провести анализ предметной области автомобильных сервисов, специализирующихся на иномарках. Сформулировать описание пользователей проектируемого приложения автомобильного сервиса, специализирующихся на иномарках,  для доступа к базе данных;
	\item спроектировать сущности базы данных и ограничения целостности автомобильных сервисов, специализирующихся на иномарках;
	\item выбрать средства реализации базы данных и приложения, в том числе выбор СУБД. Описать методы тестирования разработанного функционала и разработать тесты для проверки корректности работы приложения;
    \item провести исследование зависимости времени запроса к базе данных в присутствии индекса и без оного.
\end{enumerate}