\chapter{Технологическая часть}

В данном разделе будут выбраны средства реализации базы данных и приложения, в том числе СУБД. 
Будут описаны методы тестирования разработанного функционала и тесты для проверки корректности работы приложения.

\section{Выбор СУБД}

Наиболее популярными \cite{dbranking} являются следующие реляционными СУБД:

\begin{enumerate}
	\item Oracle;
	\item MySQL;
	\item Microsoft SQL Server;
	\item PostgreSQL.
\end{enumerate}

При выборе СУБД были выделены следующие критерии:

\begin{enumerate}
	\item Является свободным программным обеспечением \cite{reestrpo};
	\item Поддерживается наиболее популярными операционными системами \cite{osrating} (Windows, OS X, Linux);
	\item Наличие опыта работы с СУБД.
\end{enumerate}

В таблице \ref{tab:compare3} приведено сравнение СУБД по указанным критериям.

\begin{table}[H]
	\centering
	\caption{\label{tab:compare3}Сравнение существующих решений по указанным критериям (ч. 1)}
	\begin{tabular}{|l|c|c|c|}
		\hline \specialcell{СУБД} & \specialcell{1} & \specialcell{2} &
		\specialcell{3} \\\hline
		Oracle \cite{oracle} & - & + & -\\\hline
		MySQL \cite{mysql} & + & + & - \\\hline
		Microsoft SQL Server \cite{microsoft} & - & + & - \\\hline
		PostgreSQL \cite{postgre} & + & + & + \\\hline
	\end{tabular}
\end{table}

Таким образом, выбрана удовлетворяющая всем критериям PostgreSQL.

\section{Выбор средств реализации приложения}

Было реализовано приложение с монолитной архитектурой, диаграмма компонентов которого приведена на Рисунке \ref{img:components.svg}, где BI -- компонент бизнес логики, DataAccess -- компонент доступа к базе данных, CLI -- консольный интерфейс.

\myincludesvg
{components.svg} 
{f} 
{H} 
{0.9\textwidth} 
{Диаграмма компонентов}

В качестве языка программирования был выбран typescript \cite{typescript} в силу наличия соответствующих средств, обеспечивающих взаимодействия с выбранной СУБД, и опыта работы с данным языком программирования. 
Для кросс-платформленности приложения и возможности в перспективе работы на сервере использовалась среда выполнения Node.js \cite{nodejs}. 
Для обеспечения взаимодействия приложения с базой данных была выбрана Prisma ORM \cite{prismaorm}.

\section{Описание реализованных сущностей БД и ограничений целостности}

На Листингах \ref{tb1}-\ref{tb13} приведены скрипты создания необходимых таблиц на языке PL/pgSQL \cite{plpgsql}, используемого для работы с PostgreSQL, с учётом ранее описанных ограничений целостности.

\begin{lstlisting}[label=tb1, style=mystyle, caption=Создание таблицы Client и соотвествующих ограничений целостности]
	create table if not exists AutoService.Client (
	id			uuid default gen_random_uuid() primary key,
	email       text not null unique,
	fio         text not null,
	datebirth   timestamp not null check(extract(year from age(datebirth)) > 18),
	phone       text not null unique,
	password  text not null
	);
\end{lstlisting}

\begin{lstlisting}[label=tb2, style=mystyle, caption=Создание таблицы Car и соотвествующих ограничений целостности]
	create table if not exists AutoService.Car (
	vin			text primary key,
	nick       	text,
	year        integer not null,
	mark	   	text not null,
	color       text not null,
	run  		integer not null,
	owner_		uuid not null,
	foreign key(owner_) references AutoService.Client (id) on delete cascade
	);
\end{lstlisting}

\begin{lstlisting}[label=tb3, style=mystyle, caption=Создание таблицы MechanicStatus и соотвествующих ограничений целостности]
	create table if not exists AutoService.MechanicStatus (
	id			integer default 0 primary key,
	status		text not null unique
	);
\end{lstlisting}

\begin{lstlisting}[label=tb4, style=mystyle, caption=Создание таблицы Mechanic и соотвествующих ограничений целостности]
	create table if not exists AutoService.Mechanic (
	id			uuid default gen_random_uuid() primary key,
	email       text not null unique,
	fio         text not null,
	status		integer default 0,
	password 	text not null,
	foreign key (status) references AutoService.MechanicStatus (id)
	);
\end{lstlisting}

\begin{lstlisting}[label=tb5, style=mystyle, caption=Создание таблицы Service и соотвествующих ограничений целостности]
	create table if not exists AutoService.Service (
	id			uuid default gen_random_uuid() primary key,
	discription text not null,
	name        text not null unique,
	price		numeric(10,2)
	);
\end{lstlisting}

\begin{lstlisting}[label=tb6, style=mystyle, caption=Создание таблицы CanBeServed и соотвествующих ограничений целостности]
	create table if not exists AutoService.CanBeServed (
	id			uuid default gen_random_uuid() primary key,
	hours		integer not null,
	mechanic	uuid not null,
	service		uuid not null,
	foreign key (service) references AutoService.Service (id),
	foreign key (mechanic) references AutoService.Mechanic (id)
	);
\end{lstlisting}

\begin{lstlisting}[label=tb7, style=mystyle, caption=Создание таблицы Box и соотвествующих ограничений целостности]
	create table if not exists AutoService.Box (
	id			uuid default gen_random_uuid() primary key,
	number		integer not null unique
	);
\end{lstlisting}

\newpage

\begin{lstlisting}[label=tb8, style=mystyle, caption=Создание таблицы SheduleStatus и соотвествующих ограничений целостности]
	create table if not exists AutoService.SheduleStatus (
	id			integer default 0 primary key,
	status		text not null unique
	);
\end{lstlisting}

\begin{lstlisting}[label=tb9, style=mystyle, caption=Создание таблицы Shedule и соотвествующих ограничений целостности]
	create table if not exists AutoService.Shedule (
	id			uuid default gen_random_uuid() primary key,
	starttime	time not null,
	endtime		time not null,
	mechanic	uuid not null,
	box			uuid not null,
	dayweek		int not null,
	status		integer default 0,
	foreign key (box) references AutoService.Box (id) on delete cascade,
	foreign key (mechanic) references AutoService.Mechanic (id),
	foreign key (status) references AutoService.SheduleStatus
	);
\end{lstlisting}

\begin{lstlisting}[label=tb10, style=mystyle, caption=Создание таблицы TimeTable и соотвествующих ограничений целостности]
	create table if not exists AutoService.TimeTable (
	id			uuid default gen_random_uuid() primary key,
	datetime	timestamp not null,
	shedule 	uuid not null,
	foreign key (shedule) references AutoService.Shedule (id)
	);
\end{lstlisting}

\begin{lstlisting}[label=tb11, style=mystyle, caption=Создание таблицы ApplicationStatus и соотвествующих ограничений целостности]
	create table if not exists AutoService.ApplicationStatus (
	id			integer default 0 primary key,
	status		text not null unique
	);
\end{lstlisting}

\begin{lstlisting}[label=tb12, style=mystyle, caption=Создание таблицы Application и соотвествующих ограничений целостности]
	create table if not exists AutoService.Application(
	id			uuid default gen_random_uuid() primary key,
	timerecord	uuid default null,
	comment 	text default null, 
	service		uuid not null,
	status 		integer default 0 not null,
	car 		text not null,
	foreign key (status) references AutoService.ApplicationStatus (id),
	foreign key (timerecord) references AutoService.TimeTable (id) on delete cascade,
	foreign key (car) references AutoService.Car (vin) on delete cascade,
	foreign key (service) references AutoService.Service (id)
	);
\end{lstlisting}

\begin{lstlisting}[label=tb13, style=mystyle, caption=Создание таблицы Vocation и соотвествующих ограничений целостности]
	create table if not exists AutoService.Vocation (
	id			uuid default gen_random_uuid() primary key,
	start_date	timestamp not null,
	end_date	timestamp not null check (date_trunc('day', start_date) < date_trunc('day', end_date) and 
	(date_trunc('day', current_date) < date_trunc('day', start_date) or 
	date_trunc('day', current_date) > date_trunc('day', end_date)
	)),
	mechanic	uuid not null,
	foreign key (mechanic) references AutoService.Mechanic (id)
	);
\end{lstlisting}

\section{Описание реализованных триггера и функции}

На Листингах \ref{lst:trigger1.sql}-\ref{lst:trigger2.sql}  приведена реализация алгоритма триггера на добавление строки в таблицы \textbf{Client}, \textbf{Admi1n}, \textbf{Mechanic}.

\newpage

\includelistingpretty{trigger1.sql}{}{Реализация алгоритма триггера на добавление строки в таблицы Client, Admin, Mechanic(ч. 1)}

\newpage

\includelistingpretty{trigger2.sql}{}{Реализация алгоритма триггера на добавление строки в таблицы Client, Admin, Mechanic(ч. 2)}

\section{Описание реализованной ролевой модели}

На Листинге \ref{guest} приведен скрипт для создания роли гостя на языке PL/pgSQL.

\begin{lstlisting}[label=guest, style=mystyle, caption=Скрипт для создания роли гостя на языке PL/pgSQL]
	revoke all privileges on all tables in schema AutoService from public;
	grant select on AutoService.Service to public;
	grant insert on AutoService.Client to public;
\end{lstlisting}

На Листингах \ref{admin1}-\ref{admin2} приведен скрипт для создания роли администратора на языке PL/pgSQL.

\begin{lstlisting}[label=admin1, style=mystyle, caption=Скрипт для создания роли администратора на языке PL/pgSQL(ч. 1)]
	create role admin with createrole;
	grant select on AutoService.SheduleStatus to admin;
	grant select on AutoService.MechanicStatus to admin;
	grant select on AutoService.Box to admin;
	grant select on AutoService.ApplicationStatus to admin;
	grant select, insert, update, delete on AutoService.Service to admin;
	grant select, insert, update, delete on AutoService.Application to admin;
	grant select, insert, update, delete on AutoService.Car to admin;
	grant select, insert, update, delete on AutoService.TimeTable to admin;
\end{lstlisting}

\begin{lstlisting}[label=admin2, style=mystyle, caption=Скрипт для создания роли администратора на языке PL/pgSQL(ч. 2)]
	grant select, insert, update, delete on AutoService.Shedule to admin;
	grant select, insert, update, delete on AutoService.CanBeServed to admin;
	grant select, insert, update on AutoService.Admin to admin;
	grant select, insert, update on AutoService.Mechanic to admin;
	grant select, insert, update on AutoService.Client to admin;
	grant select, insert, update, delete on AutoService.TimeTable to admin;
	grant select, insert, update, delete on AutoService.Shedule to admin;
	grant select, insert, update, delete on AutoService.CanBeServed to admin;
	grant select, insert, update on AutoService.Admin to admin;
	grant select, insert, update on AutoService.Mechanic to admin;
	grant select, insert, update on AutoService.Client to admin;
\end{lstlisting}

На Листинге \ref{mechanic} приведен скрипт для создания роли механика на языке PL/pgSQL.

\newpage

\begin{lstlisting}[label=mechanic, style=mystyle, caption=Скрипт для создания роли механика на языке PL/pgSQL]
	create role mechanic;
	grant select on AutoService.Application to mechanic;
	grant select on AutoService.Shedule to mechanic;
	grant select on AutoService.CanBeServed to mechanic;
	grant select on AutoService.TimeTable to mechanic;
	grant select on AutoService.SheduleStatus to mechanic;
	grant select on AutoService.MechanicStatus to mechanic;
	grant select on AutoService.Box to mechanic;
	grant select on AutoService.ApplicationStatus to mechanic;
	grant select on AutoService.Client to mechanic;
	grant select on AutoService.Car to mechanic;
	grant update on AutoService.Application to mechanic;
	grant update on AutoService.Mechanic to mechanic;
	grant insert on AutoService.Application to mechanic;
\end{lstlisting}

На Листинге \ref{client} приведен скрипт для создания роли клиента на языке PL/pgSQL.

\begin{lstlisting}[label=client, style=mystyle, caption=Скрипт для создания роли клиента на языке PL/pgSQL]
	create role client;
	grant select on AutoService.Application to client;
	grant select on AutoService.Client to client;
	grant select on AutoService.Car to client;
	grant select on AutoService.ApplicationStatus to client;
	grant update on AutoService.Client to client;
	grant update on AutoService.Car to client;
	grant insert on AutoService.Car to client;
\end{lstlisting}

\section{Методы тестирования разработанного функционала и тесты для проверки корректности работы приложения}

В рамках разработанного приложения и базы данных предлагается модульное тестирование в качестве метода.

\subsection{Тестирование триггера и разработанной функции}

Согласно схеме алгоритма триггера, приведённой на Рисунке \ref{trigger}, в алгоритме присутствует проверка уникальности адреса электронной почты в таблицах \textbf{Client}, \textbf{Admin}, \textbf{Mechanic}. 

В Таблице \ref{tab:teststrigger1} приведены классы эквивалентности, модульные тесты и ожидаемый результат (ч. 1).

\begin{table}[H]
	\centering
	\caption{\label{tab:teststrigger1}Модульное тестирование триггера базы данных (ч. 1)}
	\begin{tabular}{|l|c|c|c|}
		\hline \specialcell{№} & \specialcell{Класс\\эквивалентности} & \specialcell{Запрос для теста} & \specialcell{Результат}\\\hline
		1 & \makecell{Запись с уникальным\\на три таблицы\\адресом добавляется\\в таблицу \textbf{Client}} & \makecell{insert into AutoService\\.Client(email, fio,\\ datebirth, phone, \\password)\\values ('veronika@k.ru', \\'Olga Kirova', '2003-06-04', \\'89267677888', '123');} & -\\\hline
		2 & \makecell{Запись с уникальным\\на три таблицы\\адресом добавляется\\в таблицу \textbf{Admin}} & \makecell{insert into AutoService\\.Admin(email, fio,\\ password)\\values ('vaeronika@k.ru',\\'Olga Kirova', '123');} & - \\\hline
		3 & \makecell{Запись с уникальным\\на три таблицы\\адресом добавляется\\в таблицу \textbf{Mechanic}} & \makecell{insert into AutoService.\\Mechanic(email, fio,\\ password)\\values ('vmronika@k.ru',\\'Olga Kirova', '123');} & - \\\hline
	\end{tabular}
\end{table}

В Таблице \ref{tab:teststrigger2} приведены классы эквивалентности, модульные тесты и ожидаемый результат (ч. 2).

\begin{table}[H]
	\centering
	\caption{\label{tab:teststrigger2}Модульное тестирование триггера базы данных (ч. 2)}
	\begin{tabular}{|l|c|c|c|}
		\hline \specialcell{№} & \specialcell{Класс\\эквивалентности} & \specialcell{Запрос для теста} & \specialcell{Результат}\\\hline
		4 & \makecell{Запись с не уникальным\\на три таблицы\\адресом добавляется\\в таблицу \textbf{Client}} & \makecell{insert into AutoService\\.Admin(email, fio,\\ password)\\values ('veronika@k.ru',\\'Olga Kirova', '123');} & \makecell{existing admin\\with\\same email} \\\hline
		5 & \makecell{Запись с не уникальным\\на три таблицы\\адресом добавляется\\в таблицу \textbf{Admin}} & \makecell{insert into AutoService\\.Client(email, fio, datebirth,\\phone, password)\\values ('vaeronika@k.ru',\\'Olga Kirova', '2003-06-04',\\'8926767088', '123');} & \makecell{existing mechanic\\with\\same email} \\\hline
		6 & \makecell{Запись с не уникальным\\на три таблицы\\адресом добавляется\\в таблицу \textbf{Mechanic}} & \makecell{insert into AutoService\\.Mechanic(email, fio,\\password)\\values ('vmronika@k.ru',\\'Olga Kirova', '123');} & \makecell{existing admin\\with\\same email} \\\hline
	\end{tabular}
\end{table}

В ходе тестирования запросы, приведённые в Таблицах \ref{tab:teststrigger1}-\ref{tab:teststrigger2}, выполнялись последовательно и был получен соответствующий результат. 

Не было обнаружено средств, благодаря которым можно было бы определить проверить покрытие тестами реализованный триггерами.

\subsection{Тесты проверки корректности работы приложения}

Для компонентов бизнес логики и доступа к базе данных, приведенных на Рисунке \ref{img:components.svg}, были разработаны модульные тесты и оценено покрытие ими разработанного функционала. Тесты были созданы с использованием пакета TS-Mocha \cite{mocha}, а покрытие было подсчитано с использованием функций пакета c8 \cite{c8}. 
Пример разработанного теста для бизнес логики приведен в \hyperref[chp:Appendix1]{Приложении А}.

\section{Описание интерфейса доступа к базе денных}

Было разработано консольное приложение. 
В приложении на каждое действие пользователя выводится или результат предыдущего действия, или сообщение системы, или меню, с помощью которого, вводя пункт меню или данные, пользователь может взаимодействовать с системой.
На Рисунке \ref{img:demo} продемонстрировано, как выглядит приложение при запуске.

\includeimage  
{demo}
{f}
{H}
{0.9\textwidth}
{Демонстрация работы приложения}

\section*{Вывод}
\addcontentsline{toc}{section}{Вывод}

В данном разделе были выбраны средства реализации базы данных и приложения, в том числе СУБД. 

Была выбрана СУБД  PostgreSQL, язык программирования typescript, среда выполнения Node.js, Prisma ORM с целью обеспечения взаимодействия приложения с базой данных.

Были описаны методы тестирования разработанного функционала и тесты для проверки корректности работы приложения.