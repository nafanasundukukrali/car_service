\chapter{Исследовательская часть}

В данном разделе будут приведены результаты исследования зависимости времени запроса к базе данных в присутствии индекса и без оного.

\section{Технические характеристики}
Ниже приведены технические характеристики устройства, на котором проводилось исследование:

\begin{itemize}
	\item операционная система: Ubuntu Mantic Minotaur;
	\item оперативная память: 16ГБ;
	\item процессор: Intel® Core™ i7-10510U × 8.
\end{itemize}

Исследование проводилось на ноутбуке, включенном в сеть электропитания. 
Во время исследования ноутбук был нагружен только встроенными приложениями окружения, окружением, а также непосредственно системой исследования. 

Исследование проводилось с использованием скрипта для командной оболочки Bash\cite{bashman}, приведённого в \hyperref[chp:Appendix2]{Приложении Б}, а также заранее подготовленных данных. 
Время выполнения запроса измерялось с использованием команды \textbf{explain} языка СУБД PostgreSQL\cite{timeman}. 

\section{Описание исследования}

Необходимо было исследовать зависимость времени запроса к базе данных в присутствии индекса и без оного. 
Исследование проводилось с добавлением индекса на столбец $datebirth$ таблицы $Client$. 
Данные для исследования были предварительно сгенерированы. 
Исследование проводилось от 0 до 190000 строк в таблице с шагом 10000 строк. 
На каждом шаге проводилось по 20 измерений, из которых выбиралось медианное значение. 
Время запроса считалось как сумма времени планирования и времени выполнения.

На Листинге \ref{query} приведен запрос, время выполнения которого в дальнейшем будет исследовано.

\newpage

\begin{lstlisting}[label=query, style=mystyle, caption=Исследуемый запрос]
	select * 
	from AutoService.Client 
	order by datebirth;
\end{lstlisting}

\section{Результаты исследования}

В Таблице \ref{tab:time} приведены замеры времени запроса в миллисекундах.

\begin{table}[H]
	\centering
	\caption{\label{tab:time}Замеры времени выполнения в миллисекундах}
	\begin{tabular}{|r|r|r|}
		\hline \specialcell{Количество строк} & \specialcell{Индекс отсутствует} &
		\specialcell{Индекс присутствует} \\\hline
		\num{0} & \num{0.90}  & \num{1.06}  \\\hline
		\num{10000} & \num{14.30}  & \num{15.05}  \\\hline
		\num{20000} & \num{41.13}  & \num{40.96}  \\\hline
		\num{30000} & \num{61.33}  & \num{34.95}  \\\hline
		\num{40000} & \num{73.00}  & \num{42.58}  \\\hline
		\num{50000} & \num{82.87}  & \num{58.56}  \\\hline
		\num{60000} & \num{95.24}  & \num{66.81}  \\\hline
		\num{70000} & \num{116.79}  & \num{82.37}  \\\hline
		\num{80000} & \num{138.73}  & \num{94.79}  \\\hline
		\num{90000} & \num{150.84}  & \num{108.33}  \\\hline
		\num{100000} & \num{166.76}  & \num{111.93}  \\\hline
		\num{110000} & \num{174.10}  & \num{118.44}  \\\hline
		\num{120000} & \num{190.82}  & \num{126.92}  \\\hline
		\num{130000} & \num{199.59}  & \num{138.71}  \\\hline
		\num{140000} & \num{212.81}  & \num{155.90}  \\\hline
		\num{150000} & \num{215.25}  & \num{173.68}  \\\hline
		\num{160000} & \num{221.87}  & \num{188.50}  \\\hline
		\num{170000} & \num{232.86}  & \num{184.94}  \\\hline
		\num{180000} & \num{248.18}  & \num{202.87}  \\\hline
		\num{190000} & \num{264.90}  & \num{226.77}  \\\hline
	\end{tabular}
\end{table}

По полученным данным были с использованием метода наименьших квадратов аппроксимированы функции зависимости времени запроса от количества строк в таблице. Аппроксимация производилась средствами библиотеки numpy языка программирования Python\cite{numpy}.
При использовании полинома второй и более степеней обнаруживается, что коэффициенты полиномов меньше $1e-8$, что позволяет ими пренебречь, на основании чего можно сделать вывод, что для аппроксимации данным методом достаточно полинома первой степени для обеих функций.

На Рисунке приведен график, отражающий медианные значения данных и зависимости времени выполнения запроса при присутствии индекса и без.

\myincludesvg
{time.svg} 
{f} 
{H} 
{0.9\textwidth} 
{График, отражающий медианные значения данных и зависимости времени выполнения запроса при присутствии индекса и без}

\section*{Вывод}
\addcontentsline{toc}{section}{Вывод}

В данном разделе были приведены результаты исследования зависимости времени запроса к базе данных в присутствии индекса и без оного.
На основании результатов исследования можно сделать вывод, что при данных технических характеристиках и данном запросе время выполнения без индекса при количестве строк в таблице больше 20000 дольше, чем при его присутствии.
