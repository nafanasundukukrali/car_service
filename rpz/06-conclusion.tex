\chapter*{ЗАКЛЮЧЕНИЕ}
\addcontentsline{toc}{chapter}{ЗАКЛЮЧЕНИЕ}

В ходе курсовой работы была разработана база данных автомобильного сервиса, специализирующегося на брендах, не оказывающих поддержку в России.

Были решены следующие задачи:

\begin{enumerate}
	\item проведен анализ предметной области автомобильных сервисов, специализирующихся на иномарках. Сформулировано описание пользователей проектируемого приложения автомобильного сервиса, специализирующихся на иномарках,  для доступа к базе данных;
	\item спроектированы сущности базы данных и ограничения целостности автомобильных сервисов, специализирующихся на иномарках;
	\item выбраны средства реализации базы данных и приложения, в том числе выбор СУБД. Описаны методы тестирования разработанного функционала и разработаны тесты для проверки корректности работы приложения;
	\item проведено исследование зависимости времени запроса к базе данных в присутствии индекса и без оного.
\end{enumerate}

Поскольку функционал приложения поделен на компоненты, в дальнейшем приложение можно будет усовершенствовать. Например, используя текущие компоненты в качестве серверной части и добавив компонент контролер,  можно, дополнительно реализовав отдельное приложение для Web UI, получить полноценное клиент-серверное приложение. Или можно реализовать вертикальный срез, добавлением функционала для работы с договорами хранения.
